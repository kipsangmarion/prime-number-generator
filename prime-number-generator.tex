\documentclass{article}
\usepackage{amsmath}
\usepackage{algorithm}
\usepackage[noend]{algpseudocode}
\usepackage{graphicx}
\usepackage{listings}

\title{Eratosthenes Sieve Algorithm for Generating Prime Numbers}
\author{Marion Kipsang}
\date{\today}

\begin{document}

\maketitle

\section{Introduction}

The Sieve of Eratosthenes is an ancient algorithm for finding all prime numbers up to a given limit. It efficiently identifies prime numbers by iteratively eliminating multiples of each prime found, leaving only the prime numbers at the end of the process.

\section{Algorithm}

The Eratosthenes Sieve algorithm can be outlined as follows:

\begin{algorithm}
\caption{Eratosthenes Sieve Algorithm}\label{eratosthenes-sieve}
\begin{algorithmic}[1]
\Procedure{SieveOfEratosthenes}{limit}
    \State Let $\text{isPrime}[0, 1, 2, \ldots, \text{limit}]$ be a boolean array initialized to \textbf{true}.
    \State $\text{isPrime}[0] \gets \text{isPrime}[1] \gets \textbf{false}$ \Comment{0 and 1 are not prime.}
    
    \For{$i \gets 2$ to $\sqrt{\text{limit}}$} \Comment{Loop through potential prime numbers.}
        \If{$\text{isPrime}[i] = \textbf{true}$} \Comment{Found a prime number.}
            \For{$j \gets i^2$ to $\text{limit}$ step $i$} \Comment{Mark multiples of $i$ as not prime.}
                \State $\text{isPrime}[j] \gets \textbf{false}$
            \EndFor
        \EndIf
    \EndFor
    
    \State \textbf{return} all indices $i$ where $\text{isPrime}[i] = \textbf{true}$
\EndProcedure
\end{algorithmic}
\end{algorithm}

\section{Explanation}

\begin{itemize}
\item The algorithm initializes a boolean array $\text{isPrime}$ with all elements set to \textbf{true}, representing that all numbers from $0$ to the \textit{limit} are initially considered prime candidates.
\item We set $\text{isPrime}[0]$ and $\text{isPrime}[1]$ to \textbf{false}, as they are not prime numbers.
\item Starting from $2$ (the first prime number), the algorithm loops through the array. If a number $i$ is found to be prime ($\text{isPrime}[i] = \textbf{true}$), all its multiples from $i^2$ up to the \textit{limit} are marked as not prime ($\text{isPrime}[j] = \textbf{false}$).
\item After the loop completes, the array will contain \textbf{true} for prime numbers and \textbf{false} for non-prime numbers.
\item The algorithm returns a list of all indices $i$ where $\text{isPrime}[i] = \textbf{true}$, which corresponds to the prime numbers up to the specified \textit{limit}.
\end{itemize}

\section{Example}

Let's apply the Eratosthenes Sieve algorithm to find all prime numbers up to $20$.

\begin{enumerate}
\item Initialize the array: $\text{isPrime} = [\textbf{true}, \textbf{true}, \textbf{true}, \ldots, \textbf{true}]$.
\item Set $\text{isPrime}[0] = \text{isPrime}[1] = \textbf{false}$.
\item Start the loop with $i = 2$. Since $\text{isPrime}[2] = \textbf{true}$, mark all multiples of $2$ as not prime: $\text{isPrime}[4] = \text{isPrime}[6] = \text{isPrime}[8] = \textbf{false}$, and so on.
\item Move to $i = 3$, which is also prime ($\text{isPrime}[3] = \textbf{true}$). Mark all multiples of $3$ as not prime: $\text{isPrime}[6] = \text{isPrime}[9] = \textbf{false}$.
\item Continue this process until $i = \sqrt{20}$.
\item At the end, the array $\text{isPrime}$ will be: $[\textbf{true}, \textbf{true}, \textbf{true}, \textbf{false}, \textbf{true}, \textbf{false}, \textbf{true}, \textbf{false}, \textbf{false}, \textbf{false}, \textbf{true}, \textbf{false}, \textbf{true}, \textbf{false}, \textbf{false}, \textbf{false}, \textbf{true}, \textbf{false}, \textbf{true}, \textbf{false}, \textbf{false}]$.
\item The prime numbers up to $20$ are: $[2, 3, 5, 7, 11, 13, 17, 19]$.
\end{enumerate}

\section{Conclusion}

The Sieve of Eratosthenes is a simple yet efficient algorithm for generating prime numbers up to a given limit. Its time complexity is $O(n \log(\log n))$, making it significantly faster than checking each number for primality individually.

\end{document}
